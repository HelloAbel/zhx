%% start of file `template-zh.tex'.
%% Copyright 2006-2013 Xavier Danaux (xdanaux@gmail.com).
%
% This work may be distributed and/or modified under the
% conditions of the LaTeX Project Public License version 1.3c,
% available at http://www.latex-project.org/lppl/.


\documentclass[11pt,a4paper,sans]{moderncv}   % possible options include font size ('10pt', '11pt' and '12pt'), paper size ('a4paper', 'letterpaper', 'a5paper', 'legalpaper', 'executivepaper' and 'landscape') and font family ('sans' and 'roman')

% moderncv 主题
\moderncvstyle{casual}                        % 选项参数是 ‘casual’, ‘classic’, ‘oldstyle’ 和 ’banking’
\moderncvcolor{blue}                          % 选项参数是 ‘blue’ (默认)、‘orange’、‘green’、‘red’、‘purple’ 和 ‘grey’
%\nopagenumbers{}                             % 消除注释以取消自动页码生成功能

% 字符编码
\usepackage[utf8]{inputenc}                   % 替换你正在使用的编码
\usepackage{CJKutf8}

% 调整页面出血
\usepackage[scale=0.75]{geometry}
%\setlength{\hintscolumnwidth}{3cm}           % 如果你希望改变日期栏的宽度

% 个人信息
\name{郗}{志红}
\title{Linux C/C++软件工程师}                     % 可选项、如不需要可删除本行

\phone[mobile]{13520140497}              % 可选项、如不需要可删除本行
\email{johanan@protonmail.com}                    % 可选项、如不需要可删除本行
%\extrainfo{附加信息 (可选项)}                 % 可选项、如不需要可删除本行
\photo[64pt][0.4pt]{picture}                  % ‘64pt’是图片必须压缩至的高度、‘0.4pt‘是图片边框的宽度 (如不需要可调节至0pt)、’picture‘ 是图片文件的名字;可选项、如不需要可删除本行


% 显示索引号;仅用于在简历中使用了引言
%\makeatletter
%\renewcommand*{\bibliographyitemlabel}{\@biblabel{\arabic{enumiv}}}
%\makeatother

% 分类索引
%\usepackage{multibib}
%\newcites{book,misc}{{Books},{Others}}
%----------------------------------------------------------------------------------
%            内容
%----------------------------------------------------------------------------------
\begin{document}
\begin{CJK}{UTF8}{gbsn}                       % 详情参阅CJK文件包
\maketitle

\section{教育背景}
\cventry{2009年 -- 2013年}{浙江海洋大学}{机械设计制造及其自动化}{本科}{\textit{}}{}  % 第3到第6编码可留白
%\cventry{年 -- 年}{学位}{院校}{城市}{\textit{成绩}}{说明}

%\section{毕业论文}
%\cvitem{题目}{\emph{题目}}
%\cvitem{导师}{导师}
%\cvitem{说明}{\small 论文简介}

\section{工作背景}
%\subsection{专业}
\cventry{2015.03 -- 至今}{软件工程师}{北京新宇合创金融软件有限公司}{}{}{%不超过1--2行的概况说明\newline{}%
工作内容:%
\begin{itemize}%
\item 对普通商家的银企直连接口进行配置与测试;
\item 对个性化商家的银企直连接口进行开发:
  \begin{itemize}%
  \item 根据商家要求的信息格式设计报文格式与内容;
  \item 根据商家的交易需求开发银企直连后台程序,使得商家的交易请求能正确完成;
%    \begin{itemize}
%    \item 三级内容 i;
%    \item 三级内容 ii;
%    \item 三级内容 iii;
%    \end{itemize}
  \item 程序开发完成后,配合商家进行程序正确性的测试与维护;
  \end{itemize}
\item 对已有的后台程序进行改进与完善。
\end{itemize}}
\cventry{2014.05 -- 2015.01}{技术员}{太原立远通信技术有限公司}{}{}{工作内容:%\newline{}说明行2
\begin{itemize}%
\item 对中国移动阳泉地区的基站与直放站进行测试与维护;
\item 对用户提出的信号问题进行解决,保证信号畅通;
\end{itemize}}
%\subsection{其他}
\cventry{2013.09 -- 2014.04}{自动化工程师}{山西华瑞机电设备有限公司}{}{}{工作内容:
\begin{itemize}%
	\item 对煤层气的自动化系统进行构图设计;
	\item 利用PLC控制语言进行上位机方面的开发;
\end{itemize}}

\section{专业技能}
\cvitemwithcomment{C语言}{熟练}{
	\begin{itemize}
	\item 熟悉关于C语言的底层开发知识;
	\item 熟练掌握C语言在Linux系统下的应用程序开发;
	\item 熟悉C在嵌入式开发中的驱动开发、裸板程序开发;
	\end{itemize}
}
\cvitemwithcomment{C++}{熟练}{
	\begin{itemize}
		\item 熟练掌握C++开发应用程序;
		\item 熟悉泛型程序开发方法;
		\item 熟悉STL的原理与合理运用;
	\end{itemize}
}
\cvitemwithcomment{Linux}{熟练}{
	\begin{itemize}
		\item 熟悉Linux操作系统的安装配置;
		\item 熟练掌握Linux系统下的系统程序与应用程序开发与调试,
		Makefile文件的编写等;
	\end{itemize}
}
\cvitemwithcomment{Python}{熟练}{
	\begin{itemize}
		\item 熟练掌握Python应用程序开发和常见标准库的使用;
		\item 熟悉Python与C/C++等外部语言的交互及扩展;
	\end{itemize}
}


%\section{计算机技能}
%\cvdoubleitem{类别 1}{XXX, YYY, ZZZ}{类别 4}{XXX, YYY, ZZZ}
%\cvdoubleitem{类别 2}{XXX, YYY, ZZZ}{类别 5}{XXX, YYY, ZZZ}
%\cvdoubleitem{类别 3}{XXX, YYY, ZZZ}{类别 6}{XXX, YYY, ZZZ}

\section{个人描述}
\cvitem{}{\small 有较强的学习欲和较丰富的软件开发经验,底层开发基础扎实,可以适应加班,}
%\cvitem{}{\small 说明}
%\cvitem{爱好 3}{\small 说明}

\section{项目经验}
\cvlistitem{项目 1:银企直连系统住建部接口
	\begin{itemize}
		\item 开发工具:Makefile、gcc、g++、tuxedo、oracle
		\item 开发平台:RedHat Linux
		\item 主要职责:主要负责开发住建部接口中定活转换、通知存款支取与查询等部分的后台程序,并为响应的程序设计正确的报文与通讯域,程序开发完成后配合各地公积金完成程序的测试并上线。
	\end{itemize}
	}
\cvlistitem{项目 2:银企直连系统东吴人寿接口
	\begin{itemize}
		\item 开发工具:Makefile、gcc、g++、tuxedo、oracle
		\item 开发平台:RedHat Linux
		\item 主要职责:主要负责开发东吴人寿接口中批量非实时代收、批量非实时代付等部分的后台程序,并为响应的程序设计正确的报文与通讯域,程序开发完成后配合东吴人寿完成程序的测试并上线。
	\end{itemize}	
	}
\cvlistitem{项目 3:银企直连系统南京国土接口
	\begin{itemize}
		\item 开发工具:Makefile、gcc、g++、tuxedo、oracle
		\item 开发平台:RedHat Linux
		\item 主要职责:主要负责开发南京国土接口中子账号生成部分的后台程序,改进完善到帐通知以及交易明细查询方面的后台程序,并为响应的程序设计正确的报文与通讯域,程序开发完成后配合南京国土完成程序的测试并上线。
	\end{itemize}	
	}

\renewcommand{\listitemsymbol}{-}             % 改变列表符号

%\section{其他 2}
%\cvlistdoubleitem{项目 1}{项目 4}
%\cvlistdoubleitem{项目 2}{项目 5\cite{book1}}
%\cvlistdoubleitem{项目 3}{}
%
% 来自BibTeX文件但不使用multibib包的出版物
%\renewcommand*{\bibliographyitemlabel}{\@biblabel{\arabic{enumiv}}}% BibTeX的数字标签
\nocite{*}
\bibliographystyle{plain}
%\bibliography{publications}                    % 'publications' 是BibTeX文件的文件名

% 来自BibTeX文件并使用multibib包的出版物
%\section{出版物}
%\nocitebook{book1,book2}
%\bibliographystylebook{plain}
%\bibliographybook{publications}               % 'publications' 是BibTeX文件的文件名
%\nocitemisc{misc1,misc2,misc3}
%\bibliographystylemisc{plain}
%\bibliographymisc{publications}               % 'publications' 是BibTeX文件的文件名

\clearpage\end{CJK}
\end{document}


%% 文件结尾 `template-zh.tex'.
